\documentclass{article}

\usepackage[utf8]{inputenc}
\usepackage[spanish]{babel}
\usepackage{fancyhdr}
\usepackage{extramarks}
\usepackage{amsmath}
\usepackage{amsthm}
\usepackage{amssymb}
\usepackage{amsfonts}
\usepackage{tikz}
\usepackage[plain]{algorithm}
\usepackage{algpseudocode}

\usetikzlibrary{automata,positioning}

%
% Basic Document Settings
%

\topmargin=-0.45in
\evensidemargin=0in
\oddsidemargin=0in
\textwidth=6.5in
\textheight=9.0in
\headsep=0.25in

\linespread{1.1}

\pagestyle{fancy}
\lhead{\hmwkAuthorName}
\chead{\hmwkClass: \hmwkTitle}
\rhead{\firstxmark}
\lfoot{\lastxmark}
\cfoot{\thepage}

\renewcommand\headrulewidth{0.4pt}
\renewcommand\footrulewidth{0.4pt}

\setlength\parindent{0pt}

%
% Create Problem Sections
%

\newcommand{\enterProblemHeader}[1]{
	\nobreak\extramarks{}{Problema \arabic{#1} continua en la siguiente pagina\ldots}\nobreak{}
	\nobreak\extramarks{Problema \arabic{#1} (continuación)}{Problema \arabic{#1} continua en la siguiente pagina\ldots}\nobreak{}
}

\newcommand{\exitProblemHeader}[1]{
	\nobreak\extramarks{Problema \arabic{#1} (continuación)}{Problema \arabic{#1} continua en la siguiente pagina\ldots}\nobreak{}
	\stepcounter{#1}
	\nobreak\extramarks{Problema \arabic{#1}}{}\nobreak{}
}

\setcounter{secnumdepth}{0}
\newcounter{partCounter}
\newcounter{homeworkProblemCounter}
\setcounter{homeworkProblemCounter}{1}
\nobreak\extramarks{Problema \arabic{homeworkProblemCounter}}{}\nobreak{}

%
% Homework Problem Environment
%
% This environment takes an optional argument. When given, it will adjust the
% problem counter. This is useful for when the problems given for your
% assignment aren't sequential. See the last 3 problems of this template for an
% example.
%
\newenvironment{homeworkProblem}[1][-1]{
	\ifnum#1>0
	\setcounter{homeworkProblemCounter}{#1}
	\fi
	\section{Problema \arabic{homeworkProblemCounter}}
	\setcounter{partCounter}{1}
	\enterProblemHeader{homeworkProblemCounter}
}{
	\exitProblemHeader{homeworkProblemCounter}
}

%
% Homework Details
%   - Title
%   - Due date
%   - Class
%   - Section/Time
%   - Instructor
%   - Author
%

\newcommand{\hmwkTitle}{Problemario\ \#1}
\newcommand{\hmwkDueDate}{\today}
\newcommand{\hmwkClass}{Matemáticas Discretas}
\newcommand{\hmwkClassTime}{}
\newcommand{\hmwkClassInstructor}{}
\newcommand{\hmwkAuthorName}{\textbf{Manuel A. Rojas}}

%
% Title Page
%

\title{
	\vspace{2in}
	\textmd{\textbf{\hmwkClass:\ \hmwkTitle}}\\
	\normalsize\vspace{0.1in}\small{\hmwkDueDate}\\
	\vspace{0.1in}\large{\textit{\hmwkClassInstructor\ \hmwkClassTime}}
	\vspace{3in}
}

\author{\hmwkAuthorName}
\date{}

\renewcommand{\part}[1]{\textbf{\large Part \Alph{partCounter}}\stepcounter{partCounter}\\}

%
% Various Helper Commands
%

% Useful for algorithms
\newcommand{\alg}[1]{\textsc{\bfseries \footnotesize #1}}

% For derivatives
\newcommand{\deriv}[1]{\frac{\mathrm{d}}{\mathrm{d}x} (#1)}

% For partial derivatives
\newcommand{\pderiv}[2]{\frac{\partial}{\partial #1} (#2)}

% Integral dx
\newcommand{\dx}{\mathrm{d}x}

% Alias for the Solution section header
\newcommand{\solution}{\textbf{\large Solution}}

% Probability commands: Expectation, Variance, Covariance, Bias
\newcommand{\E}{\mathrm{E}}
\newcommand{\Var}{\mathrm{Var}}
\newcommand{\Cov}{\mathrm{Cov}}
\newcommand{\Bias}{\mathrm{Bias}}

\begin{document}
	
\maketitle
	
\pagebreak
	
\begin{homeworkProblem}
	
%%%%%%%%%%%%%%% Problema %%%%%%%%%%%%
	
	Traducir los siguientes enunciados a lógica de predicados. El dominio sobre el que se trabajara es \(X\), el conjunto de personas. A si mismo, se pueden usar las funciones \(S(x)\), que significa ``\(x\) ha sido un estudiante del curso". \(A(x)\), que significa ``\(x\) ha obtenido una A en el curso". \(T(x)\), que significa ``\(x\) es un TA del curso". \(E(x,y)\), que significa ``\(x\) y \(y\) son la misma persona".

	\begin{enumerate}
		\item Existen personas que han tomado el curso y han obtenido una A en el curso.
		\item Todas las personas que son TA del curso y han tomado el curso y han obtenido una A.
		\item No existen personas que sean TA del curso y no hayan obtenido una A en el curso.
		\item Hay al menos tres personas que son TA del curso y no han tomado el curso.
	\end{enumerate}

%%%%%%%%%%%%%%% Solucion %%%%%%%%%%%%%%%
	
\textbf{Solución}
	
%%% PROBLEMA 1. INICIA SOLUCION %%%
	
	\begin{enumerate}
		\item \(\exists x \in X: \enspace S(x)  \land A(x)\)
		\item \(\forall x \in X: \enspace (T(x) \land S(x)) \implies A(x) \)
		\item \( \neg \exists x \in X: \enspace T(x) \land \neg A(x) \)
		\item \(\exists x, y, z \in X: \enspace (T(x) \land T(y) \land T(z) \land \neg S(x) \land \neg S(y) \land \neg S(z) \land \neg E(x, y) \land \neg E(x, z) \land \neg E(y, z) )\)		
	\end{enumerate}
%%% PROBLEMA 1. TERMINA SOLUCION %%%
	
\end{homeworkProblem}



\begin{homeworkProblem}
	
	Use tablas de verdad para comprobar el siguiente razonamiento.
		
	\begin{enumerate}
		\item \(\neg (P \lor (Q \land R)) = (\neg P) \land (\neg Q \lor \neg R) \)
		%\item \(\neg(P \land (Q \lor R)) = \neg P \lor (\neg Q \lor \neg R)\)
	\end{enumerate}

\textbf{Solución} 

%%% PROBLEMA 2. INICIA SOLUCION %%%

\begin{displaymath}  % start unumbered math environment
	%
	% Start a table in math mode.  The |c|c|c|c|c|c|c|c| string is a
	% format string that says there will be 8 colunms in the table.  The
	% c's indicate that the data in each column will be centered (use l
	% for left justified and r for right justified).  The vertical bar
	% means that lines will be drawn between columns.  The trailing
	% \hline causes a horizontal line to be drawn across the top of the
	% table.
	%
	\begin{array}{|c|c|c|c|c|c|c|c|c|c|c|}\hline
		%
		% Each row of the table consists of data separated by "&" symbols.
		% Each row must end with "\\" to cause a newline.  A trailing
		% \hline will cause a line to be drawn under the row.  A double
		% \hline is often used to separate the table header from the rest
		% of the table. 
		%
		P & Q & R & Q \land R & P \lor (Q \land R) & \neg (P \lor (Q \land R)) & \neg P & \neg Q & \neg R & \neg Q \lor \neg R & (\neg P) \land \ (\neg Q \lor \neg R) \\\hline\hline
		
		T & T & T & T & T & \mathbf{F} & F & F & F & F & \mathbf{F}\\\hline
		T & T & F & F & T & \mathbf{F} & F & F & T & T & \mathbf{F}\\\hline
		T & F & T & F & T & \mathbf{F} & F & T & F & T & \mathbf{F}\\\hline
		T & F & F & F & T & \mathbf{F} & F & T & T & T & \mathbf{F}\\\hline
		F & T & T & T & T & \mathbf{F} & T & F & F & F & \mathbf{F}\\\hline
		F & T & F & F & F & \mathbf{T} & T & F & T & T & \mathbf{T}\\\hline
		F & F & T & F & F & \mathbf{T} & T & T & F & T & \mathbf{T}\\\hline
		F & F & F & F & F & \mathbf{T} & T & T & T & T & \mathbf{T}\\\hline
	\end{array}
\end{displaymath} \\

\[\therefore \quad \neg (P \lor (Q \land R)) = (\neg P) \land (\neg Q \lor \neg R) \]

%%% PROBLEMA 2. TERMINA SOLUCION %%%

\end{homeworkProblem}

\pagebreak

\begin{homeworkProblem}
	
	\textbf{(a)} Para cada una de las siguientes expresiones, encontrar una expresión equivalente usando solamente \(\barwedge\)(nand) y \(\neg\)(neg), agrupe con paréntesis para especificar el orden en el que las operaciones se aplicaran. Puede utilizar A, B y los operadores tantas veces como desee.
	\begin{enumerate}
		\item \(A \land B\)
		\item \(A \lor B\)
		\item \(A\implies B\)
	\end{enumerate}

%%% PROBLEMA 3. INICIA SOLUCION %%%
	\textbf{Solución}
	
	Haciendo uso de tablas de verdad podemos obtener las expresiones equivalentes
	\\
	
	\textbf{Parte uno}
	
	\begin{displaymath}
		\begin{array}{|c|c|c|c|c|}\hline
			A & B & A \land B & A \barwedge B & \neg (A \barwedge B) \\\hline\hline
			T & T & \mathbf{T} & F & \mathbf{T} \\\hline
			T & F & \mathbf{F} & T & \mathbf{F} \\\hline
			F & T & \mathbf{F} & T & \mathbf{F} \\\hline
			F & F & \mathbf{F} & T & \mathbf{F} \\\hline
		\end{array}
	\end{displaymath}
	\[\therefore A \land B = \neg(A \barwedge B) \] \\
	
	\textbf{Parte dos}
	
	\begin{displaymath}
		\begin{array}{|c|c|c|c|c|c|}\hline
			A & B & \neg A & \neg B & (A \lor B) & \neg A \barwedge \neg B \\\hline\hline
	
			T & T & F & F & \mathbf{T} & \mathbf{T} \\\hline
			T & F & F & T & \mathbf{T} & \mathbf{T} \\\hline
			F & T & T & F & \mathbf{T} & \mathbf{T} \\\hline
			F & F & T & T & \mathbf{F} & \mathbf{F} \\\hline
		\end{array}
	\end{displaymath}
	\[\therefore A \lor B = \neg A \barwedge \neg B \] \\
	
	\textbf{Parte tres}
	
	\begin{displaymath}
		\begin{array}{|c|c|c|c|c|c|}\hline
			A & B & \neg A & \neg B & A \implies B & A \barwedge \neg B \\\hline\hline
			
			T & T & F & F & \mathbf{T} & \mathbf{T} \\\hline
			T & F & F & T & \mathbf{F} & \mathbf{F} \\\hline
			F & T & T & F & \mathbf{T} & \mathbf{T} \\\hline
			F & F & T & T & \mathbf{T} & \mathbf{T} \\\hline
		\end{array}
	\end{displaymath}
	\[\therefore A \implies B = A \barwedge \neg B \] \\
	
	\pagebreak

	\textbf{(b)} Es posible expresar cada expresión usando solo (nand), sin necesidad de usar la negación, Encuentre una expresión equivalente a la expresión  \(\neg A\) usando solo (nand) y agrupando con paréntesis. \\
	
	\textbf{Solución}\\
	
	\(\neg A = A \barwedge ((A \barwedge B) \barwedge B)\)
	
%%% PROBLEMA 2. TERMINA SOLUCION %%%
	
\end{homeworkProblem}

\pagebreak

\begin{homeworkProblem}
	Demuestra el siguiente enunciado demostrando su contraposición: si \(r\) es irracional, entonces \(r^\frac{1}{5}\) es irracional (Asegurese de indicar el contrapositivo explicitamente). \\
	
	\textbf{Solución}
	
	Para probar que si \(r\) es irracional, entonces \(r^\frac{1}{5}\) es irracional \(A \implies B\), usamos el enunciado logicamente equivalente, el cual es su contrapositivo, Si \(r^\frac{1}{5}\) es racional, entonces \(r\) es racional \((\neg B \implies \neg A\)). \\
	
	Ahora, sea \(x = r^\frac{1}{5}\), entonces \(r = x^5\). Ahora solo necesitamos probar que si \(x\) es racional, entonces \(x^5\) es también racional. \\
	
	Sean \(m\) y \(n\) números enteros tal que \(x = m/n  (n \neq 0)\). Entonces \(x^5\) puede expresarse en términos de \(m\) y \(n\). \(x^5 = x \cdot x \cdot x \cdot x \cdot x = \frac{5 \cdot m}{5 \cdot n}\). Siendo una fracción con un entero en el numerador y el denominador. Por lo tanto \(x^5\) es racional por definición. Lo cual queda demostrado.
	
\end{homeworkProblem}

\pagebreak

\begin{homeworkProblem}
	Suponga que \(w^{2} + x^{2} + y^{2} = z^{2}\) donde \(w,x, y\) y \(z\) denotan siempre enteros positivos (Puede resultar útil representar los enteros pares como \(2i\) y los impares como \(2j+1\) donde \(i\) y \(j\) son números enteros).

	Pruebe la siguiente proposición: \(z\) es par si y solo si \(w,x\) y \(y\) son pares. Haga esto considerando todos los casos en los que \(w,x\) y \(y\) son pares o impares. \\
	
	\textbf{Solución} 
	
	El cuadrado de un numero impar es un múltiplo de 4 mas 1:
	
	\[(2j + 1)^{2} = 4j^{2} + 4j +1 = 4(j^{2}+j)+1\]
	
	Y el cuadrado de un numero par es exactamente un múltiplo de 4:
	
	\[(2i)^{2} = 4i^{2}\]
	
		
	A si mismo, sea \(n \in 2\mathbb{Z} + 1\) (impar) \(\implies n\) es de la forma \(4k+1\) o \(4k+3\)  \\
	
	
	\textbf{Caso 1:} \(w,x,y\) son pares.
	
	\[z^{2} = w^{2} + x^{2} + y^{3} = 4a + 4b +4c = 4 (a+b+c)\]
	
	Donde \(a,b,c \in \mathbb{Z}\). \(z^{^{2}}\) es un multiplo de 4, por lo tanto es par.\\

	
	\textbf{Caso 2:} \(w,x,y\) son impares.
	
	\[z^{2} = w^{2} + x^{2} + y^{3} = (4a + 1) + (4b + 1) + (4c + 1) = 4 (a+b+c) + 3\]
	
	Donde \(a,b,c \in \mathbb{Z}\). \(z^{2}\) es impar \\
	
	\textbf{Caso 3:} \(w\) es par y \(x,y\) son impares.
	
	\[z^{2} = w^{2} + x^{2} + y^{3} = 4a + (4b+1) + (4c+1) = 4(a+ b + c) + 2\] 
	Donde \(a,b,c \in \mathbb{Z}\).
	Sin embargo el cuadrado de cualquier número no puede ser de la forma \(4k +2\) \\ 
	
	\textbf{Caso 4:} \(w, x\) son pares y \(y\) es impar.
	
	\[z^{2} = w^{2} + x^{2} + y^{3} = 4a + 4b + (4c+1) = 4(a+ b + c) + 1\] 
	Donde \(a,b,c \in \mathbb{Z}\). \(z^{2}\) es impar
	
	
	
\end{homeworkProblem}
	
\end{document}
	
